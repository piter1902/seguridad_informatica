\documentclass[10pt,a4paper]{article}
\usepackage[utf8]{inputenc}
\usepackage[spanish]{babel}
\usepackage{amsmath}
\usepackage{amsfonts}
\usepackage{amssymb}
\usepackage{graphicx}
\usepackage[left=2cm,right=2cm,top=2cm,bottom=2cm]{geometry}
\usepackage{listings}

\lstset { frame = single, breaklines = true }


\begin{document}

\begin{titlepage}
\title{\textbf{Seguridad 4}}
\author{
	Pedro Allué Tamargo (758267)
	\and
	Juan José Tambo Tambo (755742)
}
\date{\today}
\clearpage\maketitle
\thispagestyle{empty}
\tableofcontents
\end{titlepage}

\section{Identificación de vulnerabilidades}

\subsection{Hay una vulnerabilidad asociada a una variable que puede ser indexada fuera de su límite}
\begin{itemize}
\item ¿Cuál es la variable?
	\begin{itemize}
	\item La variable es \texttt{func}. Esta variable almacena un \emph{array} de punteros a funciones que devuelven \texttt{void} y no aceptan parámetros.
	\item Ejecutando el programa sin las contramedidas, cuando pide la introducción de una opción del menú, se introduce la opción 6 y se indexa el \emph{array} \texttt{funcsec}, declarado en direcciones contiguas.
	\end{itemize}
\item Indicar la línea de código que puede indexar la variable fuera de su límite.
	\begin{itemize}
	\item La variable se puede indexar fuera de su límite en la línea 131.
	\end{itemize}
\end{itemize}

\subsection{Hay vulnerabilidades de desbordamiento de búfer en el programa}
\begin{itemize}
\item ¿Cuáles son las variables?
	\begin{itemize}
	%\item La variable \texttt{resp} en la función \texttt{main}. Acepta cadenas de caracteres de hasta 512 \emph{bytes} de longitud. \textbf{(?)}
	\item La variable \texttt{comida} en la función \texttt{llenarCarrito}. Acepta 512 \emph{bytes} de longitud pero la función \texttt{scanf} no establece un límite para controlar la longitud de la cadena a copiar.
	\item La variable \texttt{malo} en la función \texttt{mostrarCalorias} utiliza una versión no segura de la función \texttt{strlen} que devuelve el número de bytes (caracteres) entre una dirección de inicio y el carácter terminador \texttt{0}. Si esta longitud es mayor que 512 (\emph{MAX\_{}SIZE}) se copiarán tantos caracteres como diga \texttt{len} o hasta llegar al carácter terminador.
	\end{itemize}
\item ¿Qué parte de la memoria asociada al proceso se puede desbordar?
	\begin{itemize}
	\item Se podría desbordar la pila. Al ser variables que se declaran en funciones y no son globales se almacenan en la pila.
	\end{itemize}
\item Indicar las líneas de código que pueden desbordar los búferes.
	\begin{itemize}
%	\item \texttt{resp}: el acceso a los datos de otro \emph{array} (\texttt{funcsec}) en la línea \textbf{nLinea}.
	\item \texttt{comida}: la función \texttt{scanf} (línea \textbf{nLinea})
	\item \texttt{malo}: la función \texttt{strlen} (línea \textbf{nLinea}) junto con la función \texttt{strncat}  (línea \textbf{nLinea}).
	\end{itemize}
\end{itemize}

\subsection{¿Hay otros tipos de vulnerabilidades en el código? ¿Cuáles?}
\begin{itemize}
\item \textbf{No tengo ni idea Hulio.}
\end{itemize}

\section{Redirección de la ejecución}
\subsection{¿Cuál es la dirección de las variables \texttt{func} y \texttt{funcsec}? ¿En qué parte de la memoria se encuentran?}

\subsection{¿Cuál es la dirección del método \texttt{showSecret1}?}

\subsection{¿Qué datos de entrada proporcionas al programa para que \texttt{func[s]} lea el puntero a la función guardado en \texttt{funcsec}, en lugar de un puntero a una función guardado en \texttt{func}?}


\section{Ejecución del método \texttt{mostrarSecreto2}}
\subsection{¿Cuál es la dirección del búfer asociado a la variable \texttt{resp}?}

\subsection{¿Qué datos de entrada proporcionas al programa para que \texttt{func[s]} lea a partir del 126º byte en \texttt{resp}, es decir, a partir de \texttt{resp[125]}?}

\subsection{¿Hay otra forma de conseguir la escritura del segundo mensaje secreto por pantalla?}


\section{Bola extra ????}




\end{document}